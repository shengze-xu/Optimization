\documentclass{article}
\usepackage{algorithm} 
\usepackage{fancyhdr}
\usepackage[toc,page]{appendix}
\usepackage{extramarks}
\usepackage{amsmath}
\usepackage{amsthm}
\usepackage{amsfonts}
\usepackage{tikz}
\usepackage{algpseudocode}
\usepackage{amssymb}
\usepackage{bbm}
\usepackage{extarrows}
\usepackage{mathrsfs}
\usepackage{Ctex}
\usepackage{dsfont}
\usepackage[hidelinks]{hyperref}
\usepackage{apacite}
\usepackage{multirow, booktabs}  
\usepackage{threeparttable}
\usepackage{dcolumn}
\usepackage{longtable}
\usepackage{threeparttablex}
\usepackage{tabu}
\usepackage{pdfpages}
\usepackage{float}
\usepackage{changepage}
\usepackage{mathtools}
\usepackage{listings}

\usetikzlibrary{automata,positioning}

\topmargin=-0.45in
\evensidemargin=0in
\oddsidemargin=0in
\textwidth=6.5in
\textheight=9.0in
\headsep=0.25in
\setlength{\parindent}{0em}
\linespread{1.1}

\pagestyle{myheadings}
\markboth{HW Solution}{Shengze Xu}
\chead{\hmwkClass\ : \hmwkTitle}
\rhead{\firstxmark}
\lfoot{\lastxmark}
\cfoot{\thepage}

\renewcommand\headrulewidth{0.4pt}
\renewcommand\footrulewidth{0.4pt}

\newcommand{\enterProblemHeader}[1]{
    \nobreak\extramarks{}{Problem \arabic{#1} continued on next page\ldots}\nobreak{}
    \nobreak\extramarks{Problem \arabic{#1} (continued)}{Problem \arabic{#1} continued on next page\ldots}\nobreak{}
}

\newcommand{\exitProblemHeader}[1]{
    \nobreak\extramarks{Problem \arabic{#1} (continued)}{Problem \arabic{#1} continued on next page\ldots}\nobreak{}
    \stepcounter{#1}
    \nobreak\extramarks{Problem \arabic{#1}}{}\nobreak{}
}

\setcounter{secnumdepth}{0}
\newcounter{partCounter}
\newcounter{homeworkProblemCounter}
\setcounter{homeworkProblemCounter}{1}
\nobreak\extramarks{Problem \arabic{homeworkProblemCounter}}{}\nobreak{}

\newenvironment{homeworkProblem}[1][-1]{
    \ifnum#1>0
        \setcounter{homeworkProblemCounter}{#1}
    \fi
    \section{Problem \arabic{homeworkProblemCounter}}
    \setcounter{partCounter}{1}
    \enterProblemHeader{homeworkProblemCounter}
}{
    \exitProblemHeader{homeworkProblemCounter}
}

\newcommand{\hmwkTitle}{第五次作业}
\newcommand{\hmwkDueDate}{\today}
\newcommand{\hmwkClass}{优化实用算法}
\newcommand{\hmwkClassInstructor}{指导老师:徐翔}
\newcommand{\hmwkAuthorName}{徐圣泽 3190102721}

\title{
    \vspace{2in}
    \textmd{\textbf{\hmwkClass:\ \hmwkTitle}}\\
    \normalsize\vspace{0.1in}\small{\hmwkDueDate }\\
    \vspace{0.1in}\large{\textit{\hmwkClassInstructor\ }}
    \vspace{3in}
}

\author{\textbf{\hmwkAuthorName}}
\date{}

\renewcommand{\part}[1]{\textbf{\large Part \Alph{partCounter}}\stepcounter{partCounter}\\}

% Various Helper Commands
% Useful for algorithms
\newcommand{\alg}[1]{\textsc{\bfseries \footnotesize #1}}
% For derivatives
\newcommand{\deriv}[1]{\frac{\mathrm{d}}{\mathrm{d}x} (#1)}
% For partial derivatives
\newcommand{\pderiv}[2]{\frac{\partial}{\partial #1} (#2)}
% Integral dx
\newcommand{\dx}{\mathrm{d}x}
% Alias for the Solution section header
\newcommand{\solution}{\textbf{\large Solution}}
% Probability commands: Expectation, Variance, Covariance, Bias
\newcommand{\E}{\mathrm{E}}
\newcommand{\Var}{\mathrm{Var}}
\newcommand{\Cov}{\mathrm{Cov}}
\newcommand{\Bias}{\mathrm{Bias}}

\lstset{
	basicstyle=\tt,%行号
	numbers=left,
	rulesepcolor=\color{red!20!green!20!blue!20},
	escapeinside=``,
	xleftmargin=2em,xrightmargin=2em, aboveskip=1em,%背景框
	framexleftmargin=1.5mm,
	frame=shadowbox,%背景色
	backgroundcolor=\color[RGB]{245,245,244},%样式
	keywordstyle=\color{blue}\bfseries,
	identifierstyle=\bf,
	numberstyle=\color[RGB]{0,192,192},
	commentstyle=\it\color[RGB]{96,96,96},
	stringstyle=\rmfamily\slshape\color[RGB]{128,0,0},%显示空格
	showstringspaces=false
}

\begin{document}
\lstset{language=MATLAB}
\maketitle
	
\pagebreak

\begin{homeworkProblem}
\textbf{1. Let $r_1(x)=x_2-x_1^2$, $r_2(x)=1-x_2$, $r(x)=(r_1(x),r_2(x))^T$, $f(x)=\frac{1}{2}[r_1(x)^2+r_2(x)^2]=\frac{1}{2}r(x)^Tr(x)$. It is known that the solution to min $f(x)$ is $x^{\star}=(1,1)^T$.}

\textbf{(a) For any initial guess x0, give the Newton algorithm of}
$$
\left\{
\begin{aligned}
&x_{k+1}=x_k+\alpha_kp_k,\\
&p_k=-[\nabla^2f(x)]\nabla f(x),\\
&\alpha_k:f(x_k+\alpha_kp_k)=\min f(x+\alpha p_k).
\end{aligned}
\right.
$$
\textbf{(b) Try to prove when $x\to x^{\star}$,}
$$
\nabla^2 f(x)\to \nabla r(x) \nabla r(x)^T.
$$

\textbf{解:}

(a)记$J(x)=\nabla r(x)^T=(\nabla r_1(x),\nabla r_2(x))^T$,则函数$f(x)$的梯度为$\nabla f(x)=r_1(x)\nabla r_1(x)+r_2(x)\nabla r_2(x)=J(x)^Tr(x)$,对应的$Hesse$矩阵为$\nabla^2 f(x)=\nabla r_1(x)\nabla r_1(x)^T+\nabla r_2(x)\nabla r_2(x)^T+r_1(x)\nabla^2r_1(x)+r_2(x)\nabla^2r_2(x)=J(x)^TJ(x)+S(x)$。

其中$$
\nabla^2r_1(x)=\left(                
\begin{array}{cc}   
	-2 & 0\\  
	0 & 0 \\  
\end{array}
\right) ,
\nabla^2r_2(x)=\left(                
\begin{array}{cc}   
	0 & 0\\  
	0 & 0 \\  
\end{array}
\right),
S(x)=\left(                
\begin{array}{cc}   
	-2r_1(x) & 0\\  
	0& 0 \\  
\end{array}
\right)
$$

该问题需要给出牛顿法,与第一次作业的算法要求基本一致,下面进行编程求解。

首先写出函数及其梯度和$Hesse$矩阵。
\begin{lstlisting}
function y=f(x)
r=[x(2)-x(1)^2;1-x(2)]; 
y=r'*r/2;
end 
\end{lstlisting}
\begin{lstlisting}
function y=Gradf(x) 
r=[x(2)-x(1)^2;1-x(2)]; 
J=[-2*x(1),1;0,-1]; 
y=J'*r;
end 
\end{lstlisting}
\begin{lstlisting}
function y=Hessef(x) 
J=[-2*x(1),1;0,-1]; 
y=J'*J+[-2*(x(2)-x(1)^2),0;0,0];
end
\end{lstlisting}

下面我们首先根据书本57至58页的算法编写了利用进退法进行一维搜索确定区间的函数:
\begin{lstlisting}
function [a,b]=Range(f,a0,h0,x,p)
t=2;k=0;ak=a0;yk=f(x+ak*p);h=h0;ak1=0;yk1=0;alpha=0;
while(0<=k)
	ak1=ak+h; 
	if(ak1<0) 
		ak1=0; 
		break;
	end
	yk1=f(x+ak1*p); 
	if(yk1<=yk)
		h=t*h; alpha=ak; ak=ak1; yk=yk1; k=k+1; 
	else
		if k==0
			h=-h; ak=ak1; yk=f(x+ak*p);
		else
			break; 
		end
	end
end
if(alpha<ak1) 
	a=alpha;
	b=ak1; 
else
	a=ak1;
	b=alpha; 
end
end
\end{lstlisting}
下面利用0.618法进行精确一维搜索:
\begin{lstlisting}
function y = Minimum(f,a0,b0,x,p,epsilon)
a=a0;b=b0; 
lambda=a+0.382*(b-a);mu=a+0.618*(b-a);
y1=f(x+lambda*p);y2=f(x+mu*p);
k=1;y=0; 
while(k<=10000)
	if(y1>y2) 
		if(b-lambda<=epsilon) 
			y=mu; 
			break; 
		end
		a=lambda; lambda=mu; 
		y1=f(x+lambda*p); 
		mu=a+0.618*(b-a); 
		y2=f(x+mu*p); 
	else
		if(mu-a<=epsilon)
			y=lambda; 
			break; 
		end
		b=mu;mu=lambda;
		y2=f(x+mu*p);
		lambda=a+0.382*(b-a); 
		y1=f(x+lambda*p); 
	end
	k=k+1; 
end
end
\end{lstlisting}
下面是带线搜索牛顿法的函数:
\begin{lstlisting}
function [x,k]=NewtonsLineSearch(f,Gradf,Hessef,x0,epsilon1,epsilon2)
k=0;x=x0;
while(k>=0) 
	if(sqrt((Gradf(x))'*Gradf(x))<=epsilon1)
		break; 
	end
	p=-inv(Hessef(x))*Gradf(x); 
	a0=2;h0=0.5;
	[a,b] = Range(f,a0,h0,x,p);
	x=x+Minimum(f,a,b,x,p,epsilon2)*p; 
	k=k+1; 
end
end
\end{lstlisting}

下面是主程序的代码:
\begin{lstlisting}
x0=unifrnd(-1000,1000,2,1); 
epsilon=10^(-12);
[x,k] = NewtonsLineSearch(@f,@Gradf,@Hessef,x0,epsilon,epsilon);
\end{lstlisting}

下面是三次随机选取初值后运行的结果:
\begin{lstlisting}
随机选定的初值:
600.5609
-716.2273

带线搜索牛顿法得到的函数极小值点:
-1.0000
1.0000

带线搜索牛顿法得到的函数极小值:
2.4850e-26
\end{lstlisting}
\begin{lstlisting}
随机选定的初值:
389.6572
-365.8010

带线搜索牛顿法得到的函数极小值点:
-1
1

带线搜索牛顿法得到的函数极小值:
0

迭代次数为:
12
\end{lstlisting}
\begin{lstlisting}
随机选定的初值:
-446.1540
-907.6572

带线搜索牛顿法得到的函数极小值点:
1.0000
1.0000

带线搜索牛顿法得到的函数极小值:
8.4926e-30

迭代次数为:
12
\end{lstlisting}

上面的运行结果说明$x^{\star}=(1,1)^T$和$x^{\star}=(-1,1)^T$均为$f(x)$的全局最小值点,此时$r_1(x^{\star})=r_2(x^{\star})=f(x^{\star})=0$,故本方法在给定初值不同的情况下给出的极小值点可能不同。

(b)由(a)知,$r_1(x^{\star})=r_2(x^{\star})=0$,且$r_i(x^{\star}),i=1,2$关于$x$连续,故$x\to 0$时$r(x)\to 0$,则有
$$
\nabla^2 f(x)=J(x)^TJ(x)+S(x)=J(x)^TJ(x)+\left(                
\begin{array}{cc}   
	-2r_1(x) & 0\\  
	0& 0 \\  
\end{array}
\right)\to J(x)^TJ(x)
$$

则有$\nabla^2 f(x)\to \nabla r(x)\nabla r(x)^T$,即证。
\end{homeworkProblem}

\begin{homeworkProblem}
\textbf{2. Let $x_1$, $x_2$ are solutions to $(A^TA+\mu_iI)x=-A^Tr,i=1,2$ with respect to $\mu_1$ and $\mu_2$, where $\mu_1>\mu_2>0,A\in R^{m\times n},r\in R^m$. Try to prove $\|Ax_2+r\|_2^2<\|Ax_1+r\|_2^2$.}

\textbf{解:}对$A^TA$应用$SVD$分解,则有正交矩阵$Q$和对角化矩阵$\Lambda$,使得$A^TA=Q\Lambda Q^T$,其中$Q=(q_1,q_2,\cdots,q_n)\in R^{n\times n}$,$\Lambda=diag(\lambda_1,\lambda_2,\cdots,\lambda_n)\in R^{n\times n}$。

根据第四次作业第三道题目的结论,我们有
$$
x(\mu_i)=-\sum_{k=1}^{n}\frac{q_k^TA^Tr}{\mu_i+\lambda_k}q_k
$$

因此有$\|Ax_i+r\|_2^2=(Ax_i+r)^T(Ax_i+r)=x_i^TA^TAx_i+x_i^TA^Tr+r^TAx_i+r^Tr$。
$$
\begin{aligned}
x_i^TA^TAx_i&=(\sum_{k=1}^{n}\frac{q_k^TA^Tr}{\mu_i+\lambda_k}q_k^T)A^TA(\sum_{k=1}^{n}\frac{q_k^TA^Tr}{\mu_i+\lambda_k}q_k)\\
&=\sum_{l=1}^n\sum_{k=1}^n(\frac{q_k^TA^Tr}{\mu_i+\lambda_k})q_k^TA^TAq_l(\frac{q_l^TA^Tr}{\mu_i+\lambda_l})
\end{aligned}
$$

易知其中$$q_k^TA^TAq_l=\left\{
\begin{aligned}
	\lambda_j,&\quad k=l=\lambda_j\\
	0,&\quad k\neq l
\end{aligned}
\right.
$$

因此$x_i^TA^TAx_i=\sum_{k=1}^n(\frac{q^TA^Tr}{\mu_i+\lambda_k})^2\lambda_k$,$x_i^TA^Tr=(r^TAx_i)^T=r_TAx_i=-\sum_{k=1}^n\frac{(r^TAq_k)(q_k^TA^Tr)}{\mu_i+\lambda_k}$,

故$\|Ax_i+r\|_2^2=\sum_{k=1}^{n}[\frac{\lambda_k(q_k^TA^Tr)^2}{(\mu_i+\lambda_k)^2}-\frac{2(q_k^TA^Tr)^2}{\mu_i+\lambda_k}]+r^Tr$。

对$\mu_i$求导,有
$$
\begin{aligned}
\frac{d}{d\mu_i}[\frac{\lambda_k(q_k^TA^Tr)^2}{(\mu_i+\lambda_k)^2}-\frac{2(q_k^TA^Tr)^2}{\mu_i+\lambda_k}]&=(q_k^TA^Tr)^2[\frac{-2\lambda_k}{(\mu_i+\lambda_k)^3}+\frac{2}{(\mu_i+\lambda_k)^2}]\\
&=(q_k^TA^Tr)^2\frac{2\mu_i}{(\mu_i+\lambda_k)^3}>0
\end{aligned}
$$

因此$\|Ax(\mu_i)+r\|_2^2$随$\mu_i$的增大而增大,因此有$\|Ax_2+r\|_2^2<\|Ax_1+r\|_2^2$。
\end{homeworkProblem}

\end{document}
