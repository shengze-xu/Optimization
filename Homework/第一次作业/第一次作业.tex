\documentclass{article}

\usepackage{fancyhdr}
\usepackage[toc,page]{appendix}
\usepackage{extramarks}
\usepackage{amsmath}
\usepackage{amsthm}
\usepackage{amsfonts}
\usepackage{tikz}
\usepackage[plain]{algorithm}
\usepackage{algpseudocode}
\usepackage{amssymb}
\usepackage{bbm}
\usepackage{extarrows}
\usepackage{mathrsfs}
\usepackage{Ctex}
\usepackage{dsfont}
\usepackage[hidelinks]{hyperref}
\usepackage{apacite}
\usepackage{multirow, booktabs}  
\usepackage{threeparttable}
\usepackage{dcolumn}
\usepackage{longtable}
\usepackage{threeparttablex}
\usepackage{tabu}
\usepackage{pdfpages}
\usepackage{float}
\usepackage{changepage}
\usepackage{mathtools}
\usepackage{listings}

\usetikzlibrary{automata,positioning}

\topmargin=-0.45in
\evensidemargin=0in
\oddsidemargin=0in
\textwidth=6.5in
\textheight=9.0in
\headsep=0.25in
\setlength{\parindent}{0em}
\linespread{1.1}

\pagestyle{myheadings}
\markboth{HW Solution}{Shengze Xu}
\chead{\hmwkClass\ : \hmwkTitle}
\rhead{\firstxmark}
\lfoot{\lastxmark}
\cfoot{\thepage}

\renewcommand\headrulewidth{0.4pt}
\renewcommand\footrulewidth{0.4pt}

\newcommand{\enterProblemHeader}[1]{
    \nobreak\extramarks{}{Problem \arabic{#1} continued on next page\ldots}\nobreak{}
    \nobreak\extramarks{Problem \arabic{#1} (continued)}{Problem \arabic{#1} continued on next page\ldots}\nobreak{}
}

\newcommand{\exitProblemHeader}[1]{
    \nobreak\extramarks{Problem \arabic{#1} (continued)}{Problem \arabic{#1} continued on next page\ldots}\nobreak{}
    \stepcounter{#1}
    \nobreak\extramarks{Problem \arabic{#1}}{}\nobreak{}
}

\setcounter{secnumdepth}{0}
\newcounter{partCounter}
\newcounter{homeworkProblemCounter}
\setcounter{homeworkProblemCounter}{1}
\nobreak\extramarks{Problem \arabic{homeworkProblemCounter}}{}\nobreak{}

\newenvironment{homeworkProblem}[1][-1]{
    \ifnum#1>0
        \setcounter{homeworkProblemCounter}{#1}
    \fi
    \section{Problem \arabic{homeworkProblemCounter}}
    \setcounter{partCounter}{1}
    \enterProblemHeader{homeworkProblemCounter}
}{
    \exitProblemHeader{homeworkProblemCounter}
}

\newcommand{\hmwkTitle}{第一次作业}
\newcommand{\hmwkDueDate}{\today}
\newcommand{\hmwkClass}{优化实用算法}
\newcommand{\hmwkClassInstructor}{指导老师:徐翔}
\newcommand{\hmwkAuthorName}{徐圣泽 3190102721}

\title{
    \vspace{2in}
    \textmd{\textbf{\hmwkClass:\ \hmwkTitle}}\\
    \normalsize\vspace{0.1in}\small{\hmwkDueDate }\\
    \vspace{0.1in}\large{\textit{\hmwkClassInstructor\ }}
    \vspace{3in}
}

\author{\textbf{\hmwkAuthorName}}
\date{}

\renewcommand{\part}[1]{\textbf{\large Part \Alph{partCounter}}\stepcounter{partCounter}\\}

% Various Helper Commands
% Useful for algorithms
\newcommand{\alg}[1]{\textsc{\bfseries \footnotesize #1}}
% For derivatives
\newcommand{\deriv}[1]{\frac{\mathrm{d}}{\mathrm{d}x} (#1)}
% For partial derivatives
\newcommand{\pderiv}[2]{\frac{\partial}{\partial #1} (#2)}
% Integral dx
\newcommand{\dx}{\mathrm{d}x}
% Alias for the Solution section header
\newcommand{\solution}{\textbf{\large Solution}}
% Probability commands: Expectation, Variance, Covariance, Bias
\newcommand{\E}{\mathrm{E}}
\newcommand{\Var}{\mathrm{Var}}
\newcommand{\Cov}{\mathrm{Cov}}
\newcommand{\Bias}{\mathrm{Bias}}

\lstset{
	basicstyle=\tt,%行号
	numbers=left,
	rulesepcolor=\color{red!20!green!20!blue!20},
	escapeinside=``,
	xleftmargin=2em,xrightmargin=2em, aboveskip=1em,%背景框
	framexleftmargin=1.5mm,
	frame=shadowbox,%背景色
	backgroundcolor=\color[RGB]{245,245,244},%样式
	keywordstyle=\color{blue}\bfseries,
	identifierstyle=\bf,
	numberstyle=\color[RGB]{0,192,192},
	commentstyle=\it\color[RGB]{96,96,96},
	stringstyle=\rmfamily\slshape\color[RGB]{128,0,0},%显示空格
	showstringspaces=false
}

\begin{document}
\lstset{language=MATLAB}
\maketitle
	
\pagebreak

\begin{homeworkProblem}
\textbf{1. Prove that $\|Bx\|\geq\|x\|/\|B^{-1}\|$, for any non singular matrix $B$.}

\textbf{证明:}当$x=0$时,$0=\|Bx\|=\|x\|/\|B^{-1}\|=0$,命题成立。

当$x\neq0$时,根据矩阵范数的定义,我们有$\|B^{-1}\|=\max_{x\neq 0}\frac{\|B^{-1}x\|}{\|x\|}$,因此我们有
$$
\|B^{-1}\|=\max_{x\neq 0}\frac{\|B^{-1}x\|}{\|x\|}\geq\frac{\|B^{-1}\cdot Bx\|}{\|Bx\|}=\frac{\|x\|}{\|Bx\|}
$$
需要注意的是,因为$B$为非奇异阵,因此$\|Bx\|\neq0$,故由上式可得$\|Bx\|\geq\|x\|/\|B^{-1}\|$。

综上所述,原命题成立。
\end{homeworkProblem}

\begin{homeworkProblem}
\textbf{2. Given a square nonsingular matrix $A$.  Consider its rank-one update $\overline{A}=A+ab^T$, where $a,b\in R^n$}

\textbf{(a) Verify that when $\overline{A}$ is nonsingular, we have}
$$
\overline{A}^{-1}=A^{-1}-\frac{A^{-1}ab^TA^{-1}}{1+b^TA^{-1}a}
$$
\textbf{证明:}
$$
\begin{aligned}
	(A^{-1}-\frac{A^{-1}ab^TA^{-1}}{1+b^TA^{-1}a})\overline{A}&=(A^{-1}-\frac{A^{-1}ab^TA^{-1}}{1+b^TA^{-1}a})(A+ab^T)\\
	&=I+\frac{A^{-1}ab^Tb^TA^{-1}a-A^{-1}ab^TA^{-1}ab^T}{1+b^TA^{-1}a}\\
	&=I+\frac{A^{-1}ab^Tb^TA^{-1}a-A^{-1}a(b^TA^{-1}a)b^T}{1+b^TA^{-1}a}\\
	&=I+\frac{A^{-1}ab^Tb^TA^{-1}a-A^{-1}ab^T(b^TA^{-1}a)}{1+b^TA^{-1}a}\\
	&=I
\end{aligned}
$$

要注意的是,上式的变换中,因为$b^TA^{-1}a$为实数,因此我们可以在式中交换其位置。

因此有$(A^{-1}-\frac{A^{-1}ab^TA^{-1}}{1+b^TA^{-1}a})\overline{A}=I$,故$\overline{A}^{-1}=A^{-1}-\frac{A^{-1}ab^TA^{-1}}{1+b^TA^{-1}a}$。

\textbf{(b) Using the above formula to show that}
$$
B_{k+1}=B_k+\frac{(y_k-B_ks_k)(y_k-B_ks_k)^T}{(y_k-B_ks_k)^Ts_k}
$$
\textbf{is the inverse of}
$$
H_{k+1}=H_k+\frac{(s_k-H_ky_k)(s_k-H_ky_k)^T}{(s_k-H_ky_k)^Ty_k}
$$
\textbf{when $H_k^{-1}=B_k$ is symmetric, $s_k=x_{k+1}-x_k$ and $y_{k+1}=\nabla f(x_{k+1})-\nabla f(x_k)$.}

\textbf{证明:}由(a)知,取$A=H^k$,$a=s_k-H_ky_k$,$b^T=\frac{(s_k-H_ky_k)^T}{(s_k-H_ky_k)^Ty_k}$,则$H_{k+1}=A+ab^T=\overline{A}$。

同时由题意,$H_k^{-1}=B_k$,$H_k=H_k^T$,$B_k=B_k^T$,我们有
$$
\begin{aligned}
	B_{k+1}&=H_{k+1}^{-1}=H_k^{-1}-\frac{H_k^{-1}\frac{(s_k-H_ky_k)(s_k-H_ky_k)^T}{(s_k-H_ky_k)^Ty_k}H_k^{-1}}{1+\frac{(s_k-H_ky_k)^TH_k^{-1}(s_k-H_ky_k)}{(s_k-H_ky_k)^Ty_k}}\\
	&=B_k-\frac{B_k(s_k-H_ky_k)(s_k-H_ky_k)^TB_k}{(s_k-H_ky_k)^Ty_k+(s_k-H_ky_k)^TB_k(s_k-H_ky_k)}\\
	&=B_k-\frac{(B_ks_k-y_k)(B_ks_k-y_k)^T}{(s_k-H_ky_k)^T(y_k+B_ks_k-y_k)}\\
	&=B_k-\frac{(y_k-B_ks_k)(y_k-B_ks_k)^T}{(s_k-H_ky_k)^TB_k^Ts_k}\\
	&=B_k-\frac{(y_k-B_ks_k)(y_k-B_ks_k)^T}{(B_ks_k-y_k)^Ts_k}\\
	&=B_k+\frac{(y_k-B_ks_k)(y_k-B_ks_k)^T}{(y_k-B_ks_k)^Ts_k}
\end{aligned}
$$

\end{homeworkProblem}
	
\begin{homeworkProblem}
\textbf{3. Minimize the Rosenbrock function $f(x)=100(x_2-x_1^2)^2+(1-x_1)^2$ and Beale function $f(x)=(1.5-x_1+x_1x_2)^2+(2.25-x_1+x_1x_2^2)^2+(2.625-x_1+x_1x_2^3)^2$ by the steepest descent method and Newtons method respectively, where $x^{(0)}=(-1.2,1)^T$.}

\textbf{解:}对于本题,我们通过编写程序求解,分别用最速下降法和牛顿法求解得到极小值点和极小值,并通过比较两种方法的迭代次数来衡量收敛速度。

首先令$f(x)=100(x_2-x_1^2)^2+(1-x_1)^2$,$g(x)=(1.5-x_1+x_1x_2)^2+(2.25-x_1+x_1x_2^2)^2+(2.625-x_1+x_1x_2^3)^2$。
\begin{lstlisting}
function y=f(x)
y=100*(x(2)-x(1)^2)^2+(1-x(1))^2; 
end 
\end{lstlisting}
\begin{lstlisting}
function y=g(x) 
y=(1.5-x(1)+x(1)*x(2))^2+(2.25-x(1)+x(1)*x(2)^2)^2
+(2.625-x(1)+x(1)*x(2)^3)^2; 
end
\end{lstlisting}
下面写出两个函数的梯度:
\begin{lstlisting}
function y=Gradf(x) 
y=zeros(2,1); 
y(1)=200*(x(1)^2-x(2))*2*x(1)+2*(x(1)-1); 
y(2)=200*(x(2)-x(1)^2); 
end 
\end{lstlisting}
\begin{lstlisting}
function y=Gradg(x) 
y=zeros(2,1); 
y(1)=2*(1.5-x(1)+x(1)*x(2))*(x(2)-1)+2*(2.25-x(1)+x(1)*x(2)^2)
*(x(2)^2-1)+2*(2.625-x(1)+x(1)*x(2)^3)*(x(2)^3-1); 
y(2)=2*(1.5-x(1)+x(1)*x(2))*x(1)+2*(2.25-x(1)+x(1)*x(2)^2)*2
*x(1)*x(2)+2*(2.625-x(1)+x(1)*x(2)^3)*3*x(1)*x(2)^2; 
end
\end{lstlisting}
写出两个函数的$Hesse$矩阵
\begin{lstlisting}
function y=Hessef(x) 
y=zeros(2,2); 
y(1,1)=1200*x(1)^2-400*x(2)+2; 
y(1,2)=-400*x(1); 
y(2,1)=-400*x(1); 
y(2,2)=200; 
end
\end{lstlisting}
\begin{lstlisting}
function y=Hesseg(x) 
y=zeros(2,2); 
y(1,1)=2*(x(2)-1)^2+2*(x(2)^2-1)^2+2*(x(2)^3-1)^2; 
y(1,2)=2*(1.5-x(1)+x(1)*x(2))+2*x(1)*(x(2)-1)
+2*(2.25-x(1)+x(1)*x(2)^2)*2*x(2)+2*2*x(1)*x(2)*(x(2)^2-1)
+2*(2.625- x(1)+x(1)*x(2)^3)*3*x(2)^2+2*3*x(1)*x(2)^2*(x(2)^3-1); 
y(2,1)=2*(1.5-x(1)+x(1)*x(2))+2*x(1)*(x(2)-1)
+2*(2.25-x(1)+x(1)*x(2)^2)*2*x(2)+2*2*x(1)*x(2)*(x(2)^2-1)
+2*(2.625- x(1)+x(1)*x(2)^3)*3*x(2)^2+2*3*x(1)*x(2)^2*(x(2)^3-1); 
y(2,2)=2*x(1)^2+2*(2*x(1)*x(2))^2+2*(2.25-x(1)
+x(1)*x(2)^2)*2*x(1)+2*(3*x(1)*x(2)^2)^2
+2*(2.625-x(1)+x(1)*x(2)^3)*6*x(1)*x(2); 
end
\end{lstlisting}
下面我们首先根据书本57至58页的算法编写了利用进退法进行一维搜索确定区间的函数:
\begin{lstlisting}
function [a,b]=Range(f,a0,h0,x,p)
t=2;k=0;ak=a0;yk=f(x+ak*p);h=h0;ak1=0;yk1=0;alpha=0;
while(0<=k)
	ak1=ak+h; 
	if(ak1<0) 
		ak1=0; 
		break;
	end
	yk1=f(x+ak1*p); 
	if(yk1<=yk)
		h=t*h; alpha=ak; ak=ak1; yk=yk1; k=k+1; 
	else
		if k==0
			h=-h; ak=ak1; yk=f(x+ak*p);
		else
			break; 
		end
	end
end
if(alpha<ak1) 
	a=alpha;
	b=ak1; 
else
	a=ak1;
	b=alpha; 
end
end
\end{lstlisting}
下面利用0.618法进行精确一维搜索:
\begin{lstlisting}
function y = Minimum(f,a0,b0,x,p,epsilon)
a=a0;b=b0; 
lambda=a+0.382*(b-a);mu=a+0.618*(b-a);
y1=f(x+lambda*p);y2=f(x+mu*p);
k=1;y=0; 
while(k<=10000)
	if(y1>y2) 
		if(b-lambda<=epsilon) 
			y=mu; 
			break; 
		end
		a=lambda; lambda=mu; 
		y1=f(x+lambda*p); 
		mu=a+0.618*(b-a); 
		y2=f(x+mu*p); 
	else
		if(mu-a<=epsilon)
			y=lambda; 
			break; 
		end
		b=mu;mu=lambda;
		y2=f(x+mu*p);
		lambda=a+0.382*(b-a); 
		y1=f(x+lambda*p); 
	end
	k=k+1; 
end
end
\end{lstlisting}
接下来我们写最速下降法和牛顿法的函数,首先写最速下降法,返回值是极小值点和迭代次数:
\begin{lstlisting}
function [x,k] = SteepestDescent(f,Gradf,x0,epsilon1,epsilon2)
while(0<=k)
	p=-Gradf(x); 
	if(sqrt(p'*p)<=epsilon1)
		break; 
	end
	a0=2;h0=0.5;
	[a,b]=Range(f,a0,h0,x,p);
	x=x+Minimum(f,a,b,x,p,epsilon2)*p; 
	k=k+1; 
end
end
\end{lstlisting}
下面是牛顿法,返回值和最速下降法相同:
\begin{lstlisting}
function [x,k] = Newtons(f,Gradf,Hessef,x0,epsilon)
k=0;x=x0;
while(0<=k) 
	p=-Gradf(x); 
	if(sqrt(p'*p)<=epsilon)
		break; 
	end
	x=x+inv(Hessef(x))*p; 
	k=k+1; 
end
end
\end{lstlisting}
下面我们取初值为$x0=[-1.2;1]^T$,利用最速下降法$SteepestDescent(f,Gradf,x_0,epsilon_1,epsilon_2)$得到Rosenbrock函数和Beale函数的极小值点和极小值:
\begin{table}[h]
	\centering
	\begin{tabular}{cccccc}
		\hline
		函数&$epsilon_1$& $epsilon_2$  & 极小值点  & 极小值  &迭代次数\\
		\hline
		Rosenbrock&$10^{-10}$& $10^{-12}$  & $(1.000,1.000)^T$  & $9.9748\times10^{-21}$  &$29239$\\
		Beale&$5\times 10^{-4}$& $10^{-9}$  & $(-55.1,1.0)^T$  & $0.48$  &$139165$\\
		\hline
	\end{tabular}
\end{table}

上面表格中$epsilon$的值在经过多次尝试后取了某合适的值进行计算,由于Beale函数以$(-1.2,1)^T$为初值的情况下并不收敛,因此取的值较大,得到近似解。

我们可以发现,在忽略误差的情况下,我们利用最速下降法得到了Rosenbrock函数的极小值点为$(1,1)^T$,极小值为$0$,符合预期,但得到的关于Beale函数的结果却不符合极小值点为$(3,0.5)^T$和极小值为$0$的预期,说明此时并未收敛向最小值点。

下面我们进行牛顿法$Newtons(f,Gradf,Hessef,x_0,epsilon)$得到Rosenbrock函数和Beale函数的极小值点和极小值:
\begin{table}[h]
	\centering
	\begin{tabular}{ccccc}
		\hline
		函数&$epsilon$ & 极小值点  & 极小值  &迭代次数\\
		\hline
		Rosenbrock&$10^{-15}$  & $(1,1)^T$  & $0$  &$7$\\
		Beale& $10^{-15}$  & $(0,1)^T $ & $1.42\times 10^{1}$  &1\\
		\hline
	\end{tabular}
\end{table}

我们发现,利用牛顿法我们成功得到了关于Rosenbrock函数的最小值点,但关于Beale函数仅仅得到了局部最小值点,并未得到全局最小值。
\end{homeworkProblem}

\begin{homeworkProblem}
\textbf{4. Let $f(x)=\frac{1}{2}x^Tx+\frac{1}{4}\sigma(x^TAx)^2$, where}
$$
\begin{bmatrix}
	5 & 1 & 0 &\frac{1}{2}\\
	1 &4 &\frac{1}{2}&0\\
	0&\frac{1}{2}&3&0\\
	\frac{1}{2}&0&0&2
\end{bmatrix}
$$
\textbf{Let (1) $x^{(0)}=(\cos70°,\sin70°,\cos70°,\sin70°)^T$;}

\textbf{(2) $x^{(0)}=(\cos50°,\sin50°,\cos50°,\sin50°)^T$}


\textbf{In the case of $\sigma=1$ and $\sigma=10^4$, discuss the numerical results and behavior of convergence rate of pure Newtons method and Newtons method with line search respectively.}
\end{homeworkProblem}

\textbf{解:}首先我们仍然定义函数及其梯度函数和$Hesse$矩阵函数,这里为方便起见,我们仍然定义$f$和$g$两个函数,但是其定义中只有$sigma$的值发生了改变,因此我们下面只以$f$相关的函数为例进行说明:
\begin{lstlisting}
function y=f(x) 
sigma=1; 
A=[5,1,0,0.5;1,4,0.5,0;0,0.5,3,0;0.5,0,0,2]; 
y=0.5*x'*x+0.25*sigma*(x'*A*x)^2; 
end
\end{lstlisting}
\begin{lstlisting}
function y=Gradf(x) 
sigma=1; 
A=[5,1,0,0.5;1,4,0.5,0;0,0.5,3,0;0.5,0,0,2]; 
y=x+sigma*(x'*A*x)*A*x; 
end
\end{lstlisting}
\begin{lstlisting}
function y=Hessef(x) 
sigma=1; 
A=[5,1,0,0.5;1,4,0.5,0;0,0.5,3,0;0.5,0,0,2]; 
y=eye(4)+2*sigma*A*x*x'*A'+sigma*sigma*(x'*A*x)*A; 
end
\end{lstlisting}
$Range$函数、$Minimum$函数、$Newtons$函数在本题仍然需要用到,但在上题中已经进行过阐述,这里不再赘述,下面是带线搜索牛顿法的函数:
\begin{lstlisting}
function [x,k]=NewtonsLineSearch(f,Gradf,Hessef,x0,epsilon1,epsilon2)
k=0;x=x0;
while(k>=0) 
	if(sqrt((Gradf(x))'*Gradf(x))<=epsilon1)
		break; 
	end
	p=-inv(Hessef(x))*Gradf(x); 
	a0=2;h0=0.3;
	[a,b] = Range(f,a0,h0,x,p);
	x=x+Minimum(f,a,b,x,p,epsilon2)*p; 
	k=k+1; 
end
end
\end{lstlisting}
\textbf{(1) $x^{(0)}=(\cos70°,\sin70°,\cos70°,\sin70°)^T$;}

下面我们利用牛顿法$Newtons(f,Gradf,Hessef,x_0,epsilon)$得到如下结果:
\begin{table}[h]
	\centering
	\begin{tabular}{ccccc}
		\hline
		$\sigma$&$epsilon$ & 极小值点  & 极小值  &迭代次数\\
		\hline
		$1$&$ 10^{-12}$  & $(9.8\times 10^{-19},7.0\times 10^{-19},1.6\times 10^{-19},1.3\times 10^{-19})^T$  & $7.5\times 10^{-39}$  &$9$\\
		$10^4$& $10^{-12}$  & $(0,0,0,0)^T $ & $0$  &$63766$\\
		\hline
	\end{tabular}
\end{table}

利用带线搜索牛顿法$NewtonsLineSearch(f,Gradf,Hessef,x_0,epsilon_1,epsilon_2)$得到如下结果:
\begin{table}[h]
	\centering
	\begin{tabular}{cccccc}
		\hline
		$\sigma$&$epsilon_1$& $epsilon_2$& 极小值点  & 极小值  &迭代次数\\
		\hline
		$1$&$ 10^{-18}$  &$ 10^{-9}$ & $(5.4\times 10^{-20},2.0\times 10^{-20},7.6\times 10^{-21},5.1\times 10^{-20})^T$  & $3.0\times 10^{-39}$  &$5$\\
		$10^4$& $10^{-18}$ &$ 10^{-9}$ & $(1.2\times 10^{-20},1.0\times 10^{-20},1.4\times 10^{-21},1.1\times 10^{-20})^T$ & $1.9\times 10^{-40}$  &$4$\\
		\hline
	\end{tabular}
\end{table}

我们可以发现,在合理的误差范围内,两种方法求得的极小值点和极小值相同,分别为$(0,0,0,0)^T$和$0$,而后者的收敛速度明显快于前者,尤其是当$\sigma$较大时,差异非常明显。

\textbf{(2) $x^{(0)}=(\cos50°,\sin50°,\cos50°,\sin50°)^T$}

下面我们利用牛顿法$Newtons(f,Gradf,Hessef,x_0,epsilon)$得到如下结果:
\begin{table}[!h]
	\centering
	\begin{tabular}{ccccc}
		\hline
		$\sigma$&$epsilon$ & 极小值点  & 极小值  &迭代次数\\
		\hline
		$1$&$ 10^{-18}$  & $(0,0,0,0)^T$  & $0$  &$10$\\
		$10^4$& $10^{-18}$  & $(1.4\times 10^{-25},9.1\times 10^{-26},1.7\times 10^{-26},1.9\times 10^{-26})^T $ & $1.5\times 10^{-50}$  &$64998$\\
		\hline
	\end{tabular}
\end{table}

利用带线搜索牛顿法$NewtonsLineSearch(f,Gradf,Hessef,x_0,epsilon_1,epsilon_2)$得到如下结果:
\begin{table}[!h]
	\centering
	\begin{tabular}{cccccc}
		\hline
		$\sigma$&$epsilon_1$& $epsilon_2$& 极小值点  & 极小值  &迭代次数\\
		\hline
		$1$&$ 10^{-20}$  &$ 10^{-10}$ & $(2.9\times 10^{-21},1.1\times 10^{-21},2.7\times 10^{-22},-3.6\times 10^{-21})^T$  & $1.1\times 10^{-41}$  &$5$\\
		$10^4$& $10^{-20}$ &$ 10^{-10}$ & $(1.3\times 10^{-21},7.4\times 10^{-22},-2.9\times 10^{-22},6.2\times 10^{-22})^T$ & $1.3\times 10^{-42}$  &$4$\\
		\hline
	\end{tabular}
\end{table}

我们可以发现,当改变初值时,两种算法得到的结果仍然没有较大差异,结论与上一种初值相比时也没有发生改变,两种方法均收敛达到了最小值点,并且在较快时间内完成了解答。

\end{document}
