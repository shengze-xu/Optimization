\documentclass{article}
\usepackage{algorithm} 
\usepackage{fancyhdr}
\usepackage[toc,page]{appendix}
\usepackage{extramarks}
\usepackage{amsmath}
\usepackage{amsthm}
\usepackage{amsfonts}
\usepackage{tikz}
\usepackage{algpseudocode}
\usepackage{amssymb}
\usepackage{bbm}
\usepackage{extarrows}
\usepackage{mathrsfs}
\usepackage{Ctex}
\usepackage{dsfont}
\usepackage[hidelinks]{hyperref}
\usepackage{apacite}
\usepackage{multirow, booktabs}  
\usepackage{threeparttable}
\usepackage{dcolumn}
\usepackage{longtable}
\usepackage{threeparttablex}
\usepackage{tabu}
\usepackage{pdfpages}
\usepackage{float}
\usepackage{changepage}
\usepackage{mathtools}
\usepackage{listings}

\usetikzlibrary{automata,positioning}

\topmargin=-0.45in
\evensidemargin=0in
\oddsidemargin=0in
\textwidth=6.5in
\textheight=9.0in
\headsep=0.25in
\setlength{\parindent}{0em}
\linespread{1.1}

\pagestyle{myheadings}
\markboth{HW Solution}{Shengze Xu}
\chead{\hmwkClass\ : \hmwkTitle}
\rhead{\firstxmark}
\lfoot{\lastxmark}
\cfoot{\thepage}

\renewcommand\headrulewidth{0.4pt}
\renewcommand\footrulewidth{0.4pt}

\newcommand{\enterProblemHeader}[1]{
    \nobreak\extramarks{}{Problem \arabic{#1} continued on next page\ldots}\nobreak{}
    \nobreak\extramarks{Problem \arabic{#1} (continued)}{Problem \arabic{#1} continued on next page\ldots}\nobreak{}
}

\newcommand{\exitProblemHeader}[1]{
    \nobreak\extramarks{Problem \arabic{#1} (continued)}{Problem \arabic{#1} continued on next page\ldots}\nobreak{}
    \stepcounter{#1}
    \nobreak\extramarks{Problem \arabic{#1}}{}\nobreak{}
}

\setcounter{secnumdepth}{0}
\newcounter{partCounter}
\newcounter{homeworkProblemCounter}
\setcounter{homeworkProblemCounter}{1}
\nobreak\extramarks{Problem \arabic{homeworkProblemCounter}}{}\nobreak{}

\newenvironment{homeworkProblem}[1][-1]{
    \ifnum#1>0
        \setcounter{homeworkProblemCounter}{#1}
    \fi
    \section{Problem \arabic{homeworkProblemCounter}}
    \setcounter{partCounter}{1}
    \enterProblemHeader{homeworkProblemCounter}
}{
    \exitProblemHeader{homeworkProblemCounter}
}

\newcommand{\hmwkTitle}{第三次作业}
\newcommand{\hmwkDueDate}{\today}
\newcommand{\hmwkClass}{优化实用算法}
\newcommand{\hmwkClassInstructor}{指导老师:徐翔}
\newcommand{\hmwkAuthorName}{徐圣泽 3190102721}

\title{
    \vspace{2in}
    \textmd{\textbf{\hmwkClass:\ \hmwkTitle}}\\
    \normalsize\vspace{0.1in}\small{\hmwkDueDate }\\
    \vspace{0.1in}\large{\textit{\hmwkClassInstructor\ }}
    \vspace{3in}
}

\author{\textbf{\hmwkAuthorName}}
\date{}

\renewcommand{\part}[1]{\textbf{\large Part \Alph{partCounter}}\stepcounter{partCounter}\\}

% Various Helper Commands
% Useful for algorithms
\newcommand{\alg}[1]{\textsc{\bfseries \footnotesize #1}}
% For derivatives
\newcommand{\deriv}[1]{\frac{\mathrm{d}}{\mathrm{d}x} (#1)}
% For partial derivatives
\newcommand{\pderiv}[2]{\frac{\partial}{\partial #1} (#2)}
% Integral dx
\newcommand{\dx}{\mathrm{d}x}
% Alias for the Solution section header
\newcommand{\solution}{\textbf{\large Solution}}
% Probability commands: Expectation, Variance, Covariance, Bias
\newcommand{\E}{\mathrm{E}}
\newcommand{\Var}{\mathrm{Var}}
\newcommand{\Cov}{\mathrm{Cov}}
\newcommand{\Bias}{\mathrm{Bias}}

\lstset{
	basicstyle=\tt,%行号
	numbers=left,
	rulesepcolor=\color{red!20!green!20!blue!20},
	escapeinside=``,
	xleftmargin=2em,xrightmargin=2em, aboveskip=1em,%背景框
	framexleftmargin=1.5mm,
	frame=shadowbox,%背景色
	backgroundcolor=\color[RGB]{245,245,244},%样式
	keywordstyle=\color{blue}\bfseries,
	identifierstyle=\bf,
	numberstyle=\color[RGB]{0,192,192},
	commentstyle=\it\color[RGB]{96,96,96},
	stringstyle=\rmfamily\slshape\color[RGB]{128,0,0},%显示空格
	showstringspaces=false
}

\begin{document}
\lstset{language=MATLAB}
\maketitle
	
\pagebreak

\begin{homeworkProblem}
\textbf{1. Try to prove that when $\phi=\phi_k^c=\frac{1}{1-\mu_k}$ where $\mu_k=\frac{(s_k^TB_ks_k)(y_k^TH_ky_k)}{(s_k^Ty_k)^2}$, the Broyden class}
$$
B_{k+1}=B_k-\frac{B_ks_ks_k^TB_k}{s_k^TB_ks_k}+\frac{y_ky_k^T}{y_k^Ts_k}+\phi_k(s_k^TB_ks_k)v_kv_k^T
$$
\textbf{where}
$$
v_k=(\frac{y_k}{y_k^Ts_k}-\frac{B_ks_k}{s_k^TB_ks_k})
$$
\textbf{becomes sigular.}

\textbf{解:}要证$B_{k+1}$为奇异矩阵,只需找到$B_{k+1}x=0$的非零解$x$。

取$x=s_k-\rho_kH_ky_k$,其中$\rho_k=\frac{y_k^Ts_k}{y_k^TH_ky_k}$且有关系$B_{k+1}s_k=y_k$和$B_kH_k=I$成立.

下面代入$x$,化简证明$B_{k+1}x=0$。
$$
\begin{aligned}
B_{k+1}x&=B_{k+1}(s_k-\rho_kH_ky_k)\\
		&=y_k-\rho_k(B_k-\frac{B_ks_ks_k^TB_k}{s_k^TB_ks_k}+\frac{y_ky_k^T}{y_k^Ts_k}+\phi_k(s_k^TB_ks_k)v_kv_k^T)H_ky_k\\
		&=(1-\rho_k)y_k+\rho_k\frac{B_ks_ks_k^Ty_k}{s_k^TB_ks_k}-\rho_k\frac{y_ky_k^TH_ky_k}{y_k^Ts_k}-\rho_k\phi_k(s_k^TB_ks_k)v_kv_k^TH_ky_k\\
		&=[(1-\rho_k)y_k^Ts_k-\rho_ky_k^TH_ky_k]\frac{y_k}{y_k^Ts_k}+\rho_ks_k^Ty_k\frac{B_ks_k}{s_k^TB_ks_k}-\rho_k\phi_k(s_k^TB_ks_k)(v_k^TH_ky_k)v_k\\
		&=-\rho_ky_k^Ts_k\frac{y_k}{y_k^Ts_k}+\rho_ks_k^Ty_k\frac{B_ks_k}{s_k^TB_ks_k}-\rho_k\phi_k(s_k^TB_ks_k)(v_k^TH_ky_k)v_k\\
		&=-\rho_ky_k^Ts_k(\frac{y_k}{y_k^Ts_k}-\frac{B_ks_k}{s_k^TB_ks_k})-\rho_k\phi_k(s_k^TB_ks_k)(v_k^TH_ky_k)v_k\\
		&=-\rho_k[y_k^Ts_k+\phi_k(s_k^TB_ks_k)(v_k^TH_ky_k)]v_k\\
		&=-\rho_k[y_k^Ts_k+\phi_k(s_k^TB_ks_k)(\frac{y_k^TH_ky_k}{y_k^Ts_k}-\frac{s_k^Ty_k}{s_k^TB_ks_k})]v_k\\
		&=-\rho_k[(1-\phi_k)s_k^Ty_k+\phi_k(s_k^TB_ks_k)\frac{y_k^TH_ky_k}{y_k^Ts_k}]v_k
\end{aligned}
$$

代入$\phi=\phi_k^c=\frac{1}{1-\mu_k}$和$\mu_k=\frac{(s_k^TB_ks_k)(y_k^TH_ky_k)}{(s_k^Ty_k)^2}$,可以得到:

$$
\begin{aligned}
B_{k+1}x&=-\rho_k\frac{-\mu_ks_k^Ty_k+(s_k^TB_ks_k)\frac{y_k^TH_ky_k}{y_k^Ts_k}}{1-\mu_k}v_k\\
		&=0
\end{aligned}
$$

因此,当$\phi=\phi_k^c=\frac{1}{1-\mu_k}$时,$B_{k+1}$为奇异矩阵。
\end{homeworkProblem}

\newpage
\begin{homeworkProblem}
\textbf{2. Using BFGS method to minimize the extended Rosenbrock function}
$$
f(x)=\sum_{i=1}^{n-1}[100(x_{i+1}-x_i^2)^2+(1-x_i)^2],
$$
\textbf{with $x_0=[-1.2,1,\cdots,-1.2,1]^T$, $x^{\star}=[1,1,\cdots,1,1]^T$ and $f(x^{\star})=0$. Try different $n=6,8,10$ and $\epsilon= 10^{-5}$. Moreover, using BFGS method to minimize the Powellsingular function}
$$
f(x)=(x_1+10x_2)^2+5(x_3-x_4)^2+(x_2-2x_3)^4+10(x_1-x_4)^4,
$$
\textbf{with $\epsilon=10^{-5}$, $x_0=[3,-1,0,1]^T$, $x^{\star}=[0,0,0,0]$ and $f(x^{\star})=0$.}

\textbf{解:}首先我们写出题目中两函数及其梯度:
\begin{lstlisting}
function y=f(x) 
n=length(x); 
y=0; 
for i=1:n-1 
	y=y+100*(x(i+1)-x(i)^2)^2+(1-x(i))^2; 
end
end
\end{lstlisting}
\begin{lstlisting}
function y=g(x) 
y=(x(1)+10*x(2))^2+5*(x(3)-x(4))^2+(x(2)-2*x(3))^4+10*(x(1)-x(4))^4; 
end
\end{lstlisting}
\begin{lstlisting}
function y=Gradf(x) 
n=length(x); 
y=zeros(n,1); 
y(1)=-400*x(1)*(x(2)-x(1)^2)-2*(1-x(1)); 
for i=2:n-1 
	y(i)=-400*x(i)*(x(i+1)-x(i)^2)-2*(1-x(i))+200*(x(i)-x(i-1)^2); 
end
y(n)=200*(x(n)-x(n-1)^2); 
end
\end{lstlisting}
\begin{lstlisting}
function y=Gradg(x) 
y=zeros(4,1); 
y(1)=2*(x(1)+10*x(2))+40*(x(1)-x(4))^3; 
y(2)=20*(x(1)+10*x(2))+4*(x(2)-2*x(3))^3; 
y(3)=10*(x(3)-x(4))-8*(x(2)-2*x(3))^3; 
y(4)=-10*(x(3)-x(4))-40*(x(1)-x(4))^3; 
end
\end{lstlisting}
\newpage
下面我们根据BFGS拟牛顿法的原理编写代码。
\begin{algorithm}[!h]  
	\caption{BFGS Method}  
	\label{alg:Framwork}  
	\begin{algorithmic}  
		\Require  
		$x_0$, $\epsilon>0$, $H_0$;
		\State $k\gets0$;
		\While{$\|\nabla f_k\|>\epsilon$}
		\State $p_k=-H_k\nabla f_k$;
		\State Compute $\alpha_k$ from a line search procedure tosatisfy the Wolfe conditions;
		\State $x_{k+1}=x_k+\alpha_k p_k$;
		\State $s_k=x_{k+1}-x_k$ and $y_k=\nabla f_{k+1}-\nabla f_k$;
		\State Compute $H_{k+1}$ by means of BFGS;
		\State $k\gets k+1$;
		\EndWhile
	\end{algorithmic}  
\end{algorithm}
\begin{lstlisting}
function [x,k] = BFGS_Method(f,Gradf,x0,epsilon)
k=0; x=x0;
n=length(x0); H=eye(n)/sqrt(Gradf(x)'*Gradf(x));
while sqrt(Gradf(x)'*Gradf(x))>epsilon 
	p=-H*Gradf(x); 
	alpha=Wolfe(f,Gradf,x,p); 
	y=Gradf(x+alpha*p)-Gradf(x); 
	x=x+alpha*p; 
	s=alpha*p; 
	H=(eye(n)-s*y'/(s'*y))*H*(eye(n)-y*s'/(s'*y))+s*s'/(s'*y); 
	k=k+1; 
end
end
\end{lstlisting}
其中我们用到了Wolfe条件下的不精确一位线搜索,调用了$Wolfe(f,gradf,x,p,max)$函数。

$Wolfe$函数的编写过程中也调用了$zoom$函数,这两个函数的算法原理和代码呈现如下。
\newpage
\begin{algorithm}[!h]  
	\caption{Line Search Algorithm for Wolfe Conditions}  
	\label{alg:Framwork}  
	\begin{algorithmic}  
		\Require  
		$\alpha_{low}$, $\alpha_{high}$;
		\State Set $\alpha_0\gets 0$, choose $\alpha_{max}>0$ and $\alpha_1\in(0,\alpha_{max})$, $i\gets 1$;
		\Repeat
		\State Evaluate $\phi(\alpha_i)$;
		\If{$\phi(\alpha_i)>\phi(0)+c_1\alpha_i\phi'(0)$ or [$\phi(\alpha_i)\geq\phi(\alpha_{i-1})$ and $i>1$]}
		\State Set $\alpha_{\star}\gets zoom(\alpha_{i-1},\alpha_i)$ and stop;
		\EndIf
		\State Evaluate $\phi'(\alpha_i)$;
		\If{$|\phi'(\alpha_i)|\leq -c_2\phi'(0)$}
		\State Set $\alpha_{\star}\gets \alpha_i$ and stop;
		\EndIf
		\If{$\phi'(\alpha_i)\geq0$ or $\phi'(\alpha_i)<c_2\phi'(0)$}
		\State Set $\alpha_{\star}\gets zoom(\alpha_{i-1},\alpha_i)$ and stop;
		\EndIf
		\State Choose $\alpha_{i+1}\in(\alpha_i,\alpha_{max})$;
		\State $i\gets i+1$;
		\Until{find out $\alpha$}
	\end{algorithmic}  
\end{algorithm}

\begin{lstlisting}
function alpha=Wolfe(f,gradf,x,p,max)
a0=0;a1=1;amax=10;c1=0.01;c2=0.4;i=1;
while i<=max
	if f(x+a1*p)>f(x)+c1*a1*gradf(x)'*p |(f(x+a1*p)>f(x+a0*p) & i>1)
		alpha=zoom(a0,a1,f,gradf,x,p); 
		break; 
	end
	if abs(gradf(x+a1*p)'*p)<=-c2*gradf(x)'*p 
		alpha=a1; 
		break; 
	end
	if gradf(x+a1*p)'*p>=0 | gradf(x+a1*p)'*p<c2*gradf(x)'*p
		alpha=zoom(a0,a1,f,gradf,x,p); 
		break; 
	end
	a0=a1;
	if a0==amax 
		alpha=a0; break; 
	end
	a1=2*a1;
	if(a1>amax) 
		a1=amax; 
	end
	i=i+1;
end
end
\end{lstlisting}

\newpage
下面我们编写zoom函数的代码。
\begin{algorithm}[!h]  
	\caption{Zoom}  
	\label{alg:Framwork}  
	\begin{algorithmic}  
		\Require  
		$\alpha_{low}$, $\alpha_{high}$;
		\Repeat
		\State Interpolate (using quadratic, cubic or bisection) to find a trial step length $\alpha_j$ between $\alpha_{low}$, $\alpha_{high}$;
		\State Evaluate $\phi(\alpha_j)$;
		\If{$\phi(\alpha_j)>\phi(0)+c_1\alpha_i\phi'(0)$ or [$\phi(\alpha_i)\geq\phi(\alpha_{low})$]}
		\State Set $\alpha_{high}\gets \alpha_j$;
		\Else
		\State Evaluate $\phi'(\alpha_i)$;
		\If{$|\phi'(\alpha_i)|\leq -c_2\phi'(0)$}
		\State Set $\alpha_{\star}\gets \alpha_j$ and stop;
		\EndIf
		\If{$\phi'(\alpha_j)(\alpha_{high}-\alpha_{low})\geq0$}
		\State Set $\alpha_{high}\gets \alpha_{low}$;
		\EndIf
		\State $\alpha_{low}\gets \alpha_j$;
		\EndIf
		\Until{find out $\alpha$}
	\end{algorithmic}  
\end{algorithm}
\begin{lstlisting}
function alpha=zoom(a_low,a_high,f,gradf,x,p)
alow=a_low;ahigh=a_high;c1=0.01;c2=0.4;
while alow>=0 
	alpha=alow+1/2*(ahigh-alow)^2*gradf(x+alow*p)'*p/(f(x+alow*p)-
	f(x+ahigh*p)+(ahigh-alow)*gradf(x+alow*p)'*p); 
	if f(x+alpha*p)>f(x)+c1*alpha*gradf(x)'*p | f(x+alpha*p)>=f(x+alow*p) 
		ahigh=alpha; 
	else
		if abs(gradf(x+alpha*p)'*p)<=-c2*gradf(x)'*p 
			break; 
		end
		if (gradf(x+alpha*p)'*p)*(ahigh-alow)>=0
			ahigh=alow; 
		end
		alow=alpha;
	end
end
end
\end{lstlisting}
\newpage
\textbf{(1) extended Rosenbrock function $f(x)=\sum_{i=1}^{n-1}[100(x_{i+1}-x_i^2)^2+(1-x_i)^2]$}

根据$x_0=[-1.2,1,\cdots,-1.2,1]^T$编写程序:
\begin{lstlisting}
x0=zeros(n,1); 
for k=1:n/2 
	x0(2*k-1)=-1.2; 
	x0(2*k)=1;
end
\end{lstlisting}

下面我们利用BFGS拟牛顿法$BFGS\_Method(f,Gradf,x0,epsilon)$函数得到如下结果:
\begin{lstlisting}
n=6时迭代次数为:52

极小值点为:
9.999999982394704e-01
9.999999990417563e-01
9.999999977048970e-01
9.999999955780986e-01
9.999999929513159e-01
9.999999871974238e-01

极小值为:
1.242386047587804e-15
\end{lstlisting}
\begin{lstlisting}
n=8时迭代次数为:66

极小值点为:
9.999999992972118e-01
1.000000000075751e+00
9.999999996216207e-01
9.999999993944044e-01
9.999999990185398e-01
9.999999985364060e-01
9.999999965513700e-01
9.999999934240992e-01

极小值为:
3.335310431010405e-16
\end{lstlisting}
\begin{lstlisting}
n=10时迭代次数为:72

极小值点为:
9.999999984675131e-01
9.999999964773540e-01
9.999999955780727e-01
9.999999920372136e-01
9.999999993778991e-01
9.999999984507908e-01
9.999999974200803e-01
9.999999991240293e-01
1.000000001821922e+00
9.999999952410858e-01

极小值为:
3.452687660445402e-14
\end{lstlisting}

\textbf{(2) Powell singular function $f(x)=(x_1+10x_2)^2+5(x_3-x_4)^2+(x_2-2x_3)^4+10(x_1-x_4)^4$}

下面我们利用BFGS拟牛顿法$BFGS\_Method(f,Gradf,x0,epsilon)$函数得到如下结果:
\begin{lstlisting}
迭代次数为:22

极小值点为:
9.040472224875362e-03
-9.040413100860243e-04
4.289113263481258e-03
4.289408586012169e-03

极小值为:
1.318008495276774e-08
\end{lstlisting}

从运行结果可以看出,BFGS方法得到的结果较为准确,且迭代次数也较少。
\end{homeworkProblem}

\end{document}
