\documentclass{article}
\usepackage{algorithm} 
\usepackage{fancyhdr}
\usepackage[toc,page]{appendix}
\usepackage{extramarks}
\usepackage{amsmath}
\usepackage{amsthm}
\usepackage{amsfonts}
\usepackage{tikz}
\usepackage{algpseudocode}
\usepackage{amssymb}
\usepackage{bbm}
\usepackage{extarrows}
\usepackage{mathrsfs}
\usepackage{Ctex}
\usepackage{dsfont}
\usepackage[hidelinks]{hyperref}
\usepackage{apacite}
\usepackage{multirow, booktabs}  
\usepackage{threeparttable}
\usepackage{dcolumn}
\usepackage{longtable}
\usepackage{threeparttablex}
\usepackage{tabu}
\usepackage{pdfpages}
\usepackage{float}
\usepackage{changepage}
\usepackage{mathtools}
\usepackage{listings}

\usetikzlibrary{automata,positioning}

\topmargin=-0.45in
\evensidemargin=0in
\oddsidemargin=0in
\textwidth=6.5in
\textheight=9.0in
\headsep=0.25in
\setlength{\parindent}{0em}
\linespread{1.1}

\pagestyle{myheadings}
\markboth{HW Solution}{Shengze Xu}
\chead{\hmwkClass\ : \hmwkTitle}
\rhead{\firstxmark}
\lfoot{\lastxmark}
\cfoot{\thepage}

\renewcommand\headrulewidth{0.4pt}
\renewcommand\footrulewidth{0.4pt}

\newcommand{\enterProblemHeader}[1]{
    \nobreak\extramarks{}{Problem \arabic{#1} continued on next page\ldots}\nobreak{}
    \nobreak\extramarks{Problem \arabic{#1} (continued)}{Problem \arabic{#1} continued on next page\ldots}\nobreak{}
}

\newcommand{\exitProblemHeader}[1]{
    \nobreak\extramarks{Problem \arabic{#1} (continued)}{Problem \arabic{#1} continued on next page\ldots}\nobreak{}
    \stepcounter{#1}
    \nobreak\extramarks{Problem \arabic{#1}}{}\nobreak{}
}

\setcounter{secnumdepth}{0}
\newcounter{partCounter}
\newcounter{homeworkProblemCounter}
\setcounter{homeworkProblemCounter}{1}
\nobreak\extramarks{Problem \arabic{homeworkProblemCounter}}{}\nobreak{}

\newenvironment{homeworkProblem}[1][-1]{
    \ifnum#1>0
        \setcounter{homeworkProblemCounter}{#1}
    \fi
    \section{Problem \arabic{homeworkProblemCounter}}
    \setcounter{partCounter}{1}
    \enterProblemHeader{homeworkProblemCounter}
}{
    \exitProblemHeader{homeworkProblemCounter}
}

\newcommand{\hmwkTitle}{第二次作业}
\newcommand{\hmwkDueDate}{\today}
\newcommand{\hmwkClass}{优化实用算法}
\newcommand{\hmwkClassInstructor}{指导老师:徐翔}
\newcommand{\hmwkAuthorName}{徐圣泽 3190102721}

\title{
    \vspace{2in}
    \textmd{\textbf{\hmwkClass:\ \hmwkTitle}}\\
    \normalsize\vspace{0.1in}\small{\hmwkDueDate }\\
    \vspace{0.1in}\large{\textit{\hmwkClassInstructor\ }}
    \vspace{3in}
}

\author{\textbf{\hmwkAuthorName}}
\date{}

\renewcommand{\part}[1]{\textbf{\large Part \Alph{partCounter}}\stepcounter{partCounter}\\}

% Various Helper Commands
% Useful for algorithms
\newcommand{\alg}[1]{\textsc{\bfseries \footnotesize #1}}
% For derivatives
\newcommand{\deriv}[1]{\frac{\mathrm{d}}{\mathrm{d}x} (#1)}
% For partial derivatives
\newcommand{\pderiv}[2]{\frac{\partial}{\partial #1} (#2)}
% Integral dx
\newcommand{\dx}{\mathrm{d}x}
% Alias for the Solution section header
\newcommand{\solution}{\textbf{\large Solution}}
% Probability commands: Expectation, Variance, Covariance, Bias
\newcommand{\E}{\mathrm{E}}
\newcommand{\Var}{\mathrm{Var}}
\newcommand{\Cov}{\mathrm{Cov}}
\newcommand{\Bias}{\mathrm{Bias}}

\lstset{
	basicstyle=\tt,%行号
	numbers=left,
	rulesepcolor=\color{red!20!green!20!blue!20},
	escapeinside=``,
	xleftmargin=2em,xrightmargin=2em, aboveskip=1em,%背景框
	framexleftmargin=1.5mm,
	frame=shadowbox,%背景色
	backgroundcolor=\color[RGB]{245,245,244},%样式
	keywordstyle=\color{blue}\bfseries,
	identifierstyle=\bf,
	numberstyle=\color[RGB]{0,192,192},
	commentstyle=\it\color[RGB]{96,96,96},
	stringstyle=\rmfamily\slshape\color[RGB]{128,0,0},%显示空格
	showstringspaces=false
}

\begin{document}
\lstset{language=PYTHON}
\maketitle
	
\pagebreak

\begin{homeworkProblem}
\textbf{1.Implement CG algorithm to solve linear systems in which $A$ is the Hilbert matrix, whose elements are $A(i,j)=\frac{1}{i+j-1}$. Set the right-hand-side to $b=(1,1,\cdots,1)^T$ and the initial point to $x_0=0$. Try dimensions $n=5,8,12,20$ and show the performance of residual with respect to iteration numbers to reduce the residual below $10^{-6}$.}

\textbf{解:}首先利用Python写出共轭梯度法的函数,代码如下:
\begin{lstlisting}
def CGmethod(A, b, n, x0, max, epsilon):
#求解Ax=b的共轭梯度法,n为维数,x0为初值,max为最大迭代次数,epsilon为精度
	k = 0
	x = x0
	r = np.dot(A, x)-b
	p = -r
	while k <= max:
		if np.sqrt(np.dot(r.T, r)) <= epsilon:
			break
		alpha = np.dot(-r.T, p)/np.dot(p.T, np.dot(A, p))
		x = x+np.array(alpha)[0][0]*p
		r0 = r
		r = r+np.dot(np.array(alpha)[0][0]*A, p)
		beta = np.dot(r.T, r)/np.dot(r0.T, r0)
		p = -r+np.array(beta)[0][0]*p
		k = k+1
	return x, k
\end{lstlisting}
下面利用该函数求解题目指定的Hilbert矩阵,代码如下:
\begin{lstlisting}
dimension=np.array([5,8,12,20])#四种维数情况
m=1
epsilon=1e-6
for index in range(4):
	n = dimension[index]
	A = [[random.random() for col in range(n)] for row in range(n)]
	x0 = [[random.random() for col in range(m)] for row in range(n)]
	b = [[random.random() for col in range(m)] for row in range(n)]
	for i in range(n):
		for j in range(n):
			A[i][j] = 1/(i+j+1)
	#A为n阶Hilbert矩阵,初始化矩阵并赋值
	for i in range(n):
		for j in range(m):
			x0[i][j] = 0
			b[i][j] = 1
	#初始化x0和b
	A = np.mat(A)
	b = np.mat(b)
	x0 = np.mat(x0)
	#将A,b,x0转化为矩阵类型
	(x, k) = CGmethod(A, b, n, x0, 10000, epsilon)
	print(x)
	print(k)
\end{lstlisting}

5阶Hilbert矩阵使用CG法,迭代次数为6,解如下:
\begin{lstlisting}
[[    4.99999996]
[ -119.99999998]
[  630.00000003]
[-1119.99999997]
[  630.00000003]]
\end{lstlisting}
8阶Hilbert矩阵使用CG法,迭代次数为20,解如下:
\begin{lstlisting}
[[-8.00001170e+00]
[ 5.04000300e+02]
[-7.56000196e+03]
[ 4.62000058e+04]
[-1.38600011e+05]
[ 2.16216014e+05]
[-1.68168011e+05]
[ 5.14800039e+04]]
\end{lstlisting}
12阶Hilbert矩阵使用CG法,迭代次数为44,解如下:
\begin{lstlisting}
[[-9.60760022e+00]
[ 8.15348107e+02]
[-1.64961591e+04]
[ 1.35509221e+05]
[-5.36480720e+05]
[ 1.02540139e+06]
[-6.42580101e+05]
[-6.57592000e+05]
[ 8.04244085e+05]
[ 6.63073788e+05]
[-1.24127915e+06]
[ 4.65505726e+05]]
\end{lstlisting}
20阶Hilbert矩阵使用CG法,迭代次数为85,解如下:
\begin{lstlisting}
[[-1.09717066e+01]
[ 1.05074723e+03]
[-2.39540259e+04]
[ 2.20415205e+05]
[-9.65326279e+05]
[ 1.99009026e+06]
[-1.25270412e+06]
[-1.34347342e+06]
[ 8.83234198e+05]
[ 1.68796915e+06]
[ 3.88214600e+05]
[-1.30552508e+06]
[-1.71054478e+06]
[-5.28252009e+05]
[ 1.20867559e+06]
[ 2.00288477e+06]
[ 9.44593867e+05]
[-1.43403647e+06]
[-2.65094537e+06]
[ 1.88784341e+06]]
\end{lstlisting}
我们发现,本题中$n$阶的Hilbert矩阵在求解时迭代次数均大于$n$,且随着维数的上升迭代次数增长非常迅速,这与一般对$n$阶矩阵应用共轭梯度法求解时迭代次数不超过$n$的结论不符,我们推测这一点可能与Hilbert矩阵的条件数有关。

经过计算发现,5阶矩阵的条件数为$4.77\times10^5$,8阶矩阵的条件数为$1.53\times 10^{10}$,12阶矩阵的条件数为$1.62\times10^{16}$,20阶矩阵的条件数为$2.11\times10^{18}$。Hilbert矩阵的条件数较大,因此共轭梯度法迭代时无法达到预想的准确程度,则CG法的收敛速度无法达到理论预期。

\end{homeworkProblem}

\begin{homeworkProblem}
\textbf{2. Derive Preconditioned CG Algorithm by applying the standard CG method in the variables $\hat{x}$ and transforming back into the original variables $x$ to see the expression of preconditioner $M$.}

\textbf{解:}对变量$x$乘上非奇异矩阵$C$得到新变量$\hat{x}$,即$\hat{x}=Cx$,此时目标函数表达式如下
$$
\phi(\hat{x})=\frac{1}{2}\hat{x}^T(C^{-T}AC^{-1})^{-1}\hat{x}-(C^{-1}b)^T\hat{x}
$$
记$\hat{A}=C^{-T}AC^{-1}$,$\hat{b}=C^{-T}b$,首先写出对于$\hat{x}$的CG算法。

\begin{algorithm}[!h]  
	\caption{CG}  
	\label{alg:Framwork}  
	\begin{algorithmic}  
		\Require  
		$\hat{x_0}=Cx_0$;
		\State $\hat{r_0}\gets C^{-T}(Ax_0-b)$, $\hat{p_0}\gets -\hat{r_0}$, $k\gets0$;
		\While{$r_k\neq 0$}
		\State $\hat{\alpha_k}\gets-\frac{\hat{r_k}^T\hat{p_k}}{\hat{p_k}^T\hat{A}\hat{p_k}}$;
		\State $\hat{x}_{k+1}\gets \hat{x_k}+\hat{\alpha_k}\hat{p_k}$;
		\State $\hat{r}_{k+1}\gets \hat{r_k}+\hat{\alpha_k}\hat{A}\hat{p_k}$;
		\State $\hat{\beta}_{k+1}\gets \frac{\hat{r}_{k+1}^T\hat{r}_{k+1}}{\hat{r_k}^T\hat{r_k}}$;
		\State $\hat{p}_{k+1}\gets -\hat{r}_{k+1}+\hat{\beta}_{k+1}\hat{p_k}$;
		\State $k\gets k+1$;
		\EndWhile
	\end{algorithmic}  
\end{algorithm}

以上就是对$\hat{x}$应用标准CG算法的流程,下面我们对上述算法的每一步进行分析,改为对$x$进行过预处理后的形式。
\begin{itemize}
\item 首先由$\hat{x_0}=Cx_0$合理推测每一步均有$\hat{x_k}=Cx_k$,同理由$\hat{r_0}=C^{-T}(Ax_0-b)=C^{-T}r_0$可知每一步也均有$\hat{r_k}=C^{-T}r_k$。
\item 
对于$\hat{x}_{k+1}\gets \hat{x_k}+\hat{\alpha_k}\hat{p_k}$,代入后可以得到$x_{k+1}\gets x_k+C^{-1}\hat{\alpha_k}\hat{p_k}$,为了保证算法的形式不变,新算法的形式也应为$x_{k+1}\gets x_k+\alpha_kp_k$,因此有$\alpha_k=\hat{\alpha_k}$,$p_k=C^{-1}\hat{p_k}$。
\item
对于$\alpha_k=\hat{\alpha_k}$,代入化简$\hat{\alpha_k}=-\frac{\hat{r_k}^T\hat{p_k}}{\hat{p_k}^T\hat{A}\hat{p_k}}$后,得到$\alpha_k=-\frac{r_k^TC^{-1}Cp_k}{p_k^TC^TC^{-T}AC^{-1}Cp_k}=-\frac{r_k^Tp_k}{p_k^TAp_k}$
\item
对于$\hat{r}_{k+1}\gets\hat{r_k}+\hat{\alpha_k}\hat{A}\hat{p_k}$,代入得到$C^{-T}r_{k+1}=C^{-T}r_k+\alpha_kC^{-T}AC^{-1}Cp_k$,化简得到$r_{k+1}=r_k+\alpha_kAp_k$。
\item
对于$\hat{p_0}=-\hat{r_0}$,代入得到$Cp_0=-C^{-T}r_0$,化简得到$p_0=-(C^TC)^{-1}r_0$,引入新变量$y_0=(C^TC)^{-1}r_0$。设预处理器$M=C^TC$,则有$y_0=M^{-1}r_0$,$p_0=-y_0$。
\item
对于$\hat{\beta}_{k+1}\gets\frac{\hat{r}_{k+1}^T\hat{r}_{k+1}}{\hat{r_k}^T\hat{r_k}}$,保持$\beta_{k+1}=\hat{\beta}_{k+1}$,则有$\beta_{k+1}=\frac{r_{k+1}^TC^{-1}C^{-T}r_{k+1}}{r_k^TC^{-1}C^{-T}r_k}=\frac{r_{k+1}^TM^{-1}r_{k+1}}{r_k^TM^{-1}r_k}=\frac{r_{k+1}^Ty_{k+1}}{r_k^Ty_k}$。
\item
对于$\hat{p}_{k+1}\gets-\hat{r}_{k+1}+\hat{\beta}_{k+1}\hat{p_k}$,代入得到$Cp_{k+1}=-C^{-T}r_{k+1}+\beta_{k+1}Cp_k$,化简得到$p_{k+1}=-M^{-1}r_{k+1}+\beta_{k+1}p_k=-y_{k+1}+\beta_{k+1}p_k$。

\end{itemize}

\begin{algorithm}[!h]  
	\caption{Preconditioned CG}  
	\label{alg:Framwork}  
	\begin{algorithmic}  
		\Require  
		$x_0$, preconditioner $M$;
		\State $r_0\gets Ax_0-b$;
		\State Solve $My_0=r_0$, $p_0\gets -y_0$, $k\gets0$;
		\While{$r_k\neq 0$}
		\State $\alpha_k\gets -\frac{r_k^Tp_k}{p_k^TAp_k}$;
		\State $x_{k+1}\gets x_k+\alpha_kp_k$;
		\State $r_{k+1}\gets r_k+\alpha_kAp_k$;
		\State Solve $My_{k+1}=r_{k+1}$;
		\State $\beta_{k+1}\gets \frac{r_{k+1}^Ty_{k+1}}{r_k^Ty_k}$;
		\State $p_{k+1}\gets -y_{k+1}+\beta_{k+1}p_k$;
		\State $k\gets k+1$;
		\EndWhile
	\end{algorithmic}  
\end{algorithm}
上面算法中的$M$为预处理器,$M=C^TC$,是一个对称正定矩阵,以上就是求解$x$的经过$M$预处理的CG算法的流程。
\end{homeworkProblem}

\end{document}
