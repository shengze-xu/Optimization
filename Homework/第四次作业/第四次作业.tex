\documentclass{article}
\usepackage{algorithm} 
\usepackage{fancyhdr}
\usepackage[toc,page]{appendix}
\usepackage{extramarks}
\usepackage{amsmath}
\usepackage{amsthm}
\usepackage{amsfonts}
\usepackage{tikz}
\usepackage{algpseudocode}
\usepackage{amssymb}
\usepackage{bbm}
\usepackage{extarrows}
\usepackage{mathrsfs}
\usepackage{Ctex}
\usepackage{dsfont}
\usepackage[hidelinks]{hyperref}
\usepackage{apacite}
\usepackage{multirow, booktabs}  
\usepackage{threeparttable}
\usepackage{dcolumn}
\usepackage{longtable}
\usepackage{threeparttablex}
\usepackage{tabu}
\usepackage{pdfpages}
\usepackage{float}
\usepackage{changepage}
\usepackage{mathtools}
\usepackage{listings}

\usetikzlibrary{automata,positioning}

\topmargin=-0.45in
\evensidemargin=0in
\oddsidemargin=0in
\textwidth=6.5in
\textheight=9.0in
\headsep=0.25in
\setlength{\parindent}{0em}
\linespread{1.1}

\pagestyle{myheadings}
\markboth{HW Solution}{Shengze Xu}
\chead{\hmwkClass\ : \hmwkTitle}
\rhead{\firstxmark}
\lfoot{\lastxmark}
\cfoot{\thepage}

\renewcommand\headrulewidth{0.4pt}
\renewcommand\footrulewidth{0.4pt}

\newcommand{\enterProblemHeader}[1]{
    \nobreak\extramarks{}{Problem \arabic{#1} continued on next page\ldots}\nobreak{}
    \nobreak\extramarks{Problem \arabic{#1} (continued)}{Problem \arabic{#1} continued on next page\ldots}\nobreak{}
}

\newcommand{\exitProblemHeader}[1]{
    \nobreak\extramarks{Problem \arabic{#1} (continued)}{Problem \arabic{#1} continued on next page\ldots}\nobreak{}
    \stepcounter{#1}
    \nobreak\extramarks{Problem \arabic{#1}}{}\nobreak{}
}

\setcounter{secnumdepth}{0}
\newcounter{partCounter}
\newcounter{homeworkProblemCounter}
\setcounter{homeworkProblemCounter}{1}
\nobreak\extramarks{Problem \arabic{homeworkProblemCounter}}{}\nobreak{}

\newenvironment{homeworkProblem}[1][-1]{
    \ifnum#1>0
        \setcounter{homeworkProblemCounter}{#1}
    \fi
    \section{Problem \arabic{homeworkProblemCounter}}
    \setcounter{partCounter}{1}
    \enterProblemHeader{homeworkProblemCounter}
}{
    \exitProblemHeader{homeworkProblemCounter}
}

\newcommand{\hmwkTitle}{第四次作业}
\newcommand{\hmwkDueDate}{\today}
\newcommand{\hmwkClass}{优化实用算法}
\newcommand{\hmwkClassInstructor}{指导老师:徐翔}
\newcommand{\hmwkAuthorName}{徐圣泽 3190102721}

\title{
    \vspace{2in}
    \textmd{\textbf{\hmwkClass:\ \hmwkTitle}}\\
    \normalsize\vspace{0.1in}\small{\hmwkDueDate }\\
    \vspace{0.1in}\large{\textit{\hmwkClassInstructor\ }}
    \vspace{3in}
}

\author{\textbf{\hmwkAuthorName}}
\date{}

\renewcommand{\part}[1]{\textbf{\large Part \Alph{partCounter}}\stepcounter{partCounter}\\}

% Various Helper Commands
% Useful for algorithms
\newcommand{\alg}[1]{\textsc{\bfseries \footnotesize #1}}
% For derivatives
\newcommand{\deriv}[1]{\frac{\mathrm{d}}{\mathrm{d}x} (#1)}
% For partial derivatives
\newcommand{\pderiv}[2]{\frac{\partial}{\partial #1} (#2)}
% Integral dx
\newcommand{\dx}{\mathrm{d}x}
% Alias for the Solution section header
\newcommand{\solution}{\textbf{\large Solution}}
% Probability commands: Expectation, Variance, Covariance, Bias
\newcommand{\E}{\mathrm{E}}
\newcommand{\Var}{\mathrm{Var}}
\newcommand{\Cov}{\mathrm{Cov}}
\newcommand{\Bias}{\mathrm{Bias}}

\lstset{
	basicstyle=\tt,%行号
	numbers=left,
	rulesepcolor=\color{red!20!green!20!blue!20},
	escapeinside=``,
	xleftmargin=2em,xrightmargin=2em, aboveskip=1em,%背景框
	framexleftmargin=1.5mm,
	frame=shadowbox,%背景色
	backgroundcolor=\color[RGB]{245,245,244},%样式
	keywordstyle=\color{blue}\bfseries,
	identifierstyle=\bf,
	numberstyle=\color[RGB]{0,192,192},
	commentstyle=\it\color[RGB]{96,96,96},
	stringstyle=\rmfamily\slshape\color[RGB]{128,0,0},%显示空格
	showstringspaces=false
}

\begin{document}
\lstset{language=MATLAB}
\maketitle
	
\pagebreak

\begin{homeworkProblem}
\textbf{1. Let $B_k\in R^{n\times n}$ symmetric positive definite. $p_k$ solves min $q_k(p)=\frac{1}{2}p^TB_kp+\nabla f(x_k)^Tp+f(x_k)$. Try to prove $p_k$ is a decent direction of $f$ at $x_k$.}

\textbf{解:}根据条件我们有
$$
B_k^Tp_k+\nabla f(x_k)=0
$$

其中$B_k$可逆,故可以解得$p_k=-B_k^{-1}\nabla f(x_k)$

代入下式得到
$$
\begin{aligned}
f(x_k+\alpha p_k)&=f(x_k)+\alpha \nabla f(x_k)^Tp_k+\alpha o(\|p_k\|)\\
&=f(x_k)+\alpha\nabla f(x_k)^T(-B_k^{-1}\nabla f(x_k))+\alpha o(\|p_k\|)\\
&=f(x_k)-\alpha\nabla f(x_k)^TB_k^{-1}\nabla f(x_k)+\alpha o(\|p_k\|)
\end{aligned}
$$

因此有
$$
f(x_k+\alpha p_k)-f(x_k)=-\alpha\nabla f(x_k)^TB_k^{-1}\nabla f(x_k)+\alpha o(\|p_k\|)
$$

上式中$\nabla f(x_k)^TB_k^{-1}\nabla f(x_k)>0$,$B_k^{-1}$是对称正定矩阵,则当$\alpha\to 0$时,$f(x_k+\alpha p_k)-f(x_k)<0$,因此$p_k$是$f$在$x_k$的下降方向。
\end{homeworkProblem}

\begin{homeworkProblem}
\textbf{2. Let $q_k(p)=\frac{1}{2}p^TB_kp+\nabla f(x_k)^Tp+f(x_k)$. Try to prove the Cauchy point is the minimizer of $q_k(p)$ along $\nabla f(x)$}
$$
p_k^C=-\tau_k\frac{\Delta_k}{\|\nabla f(x_k)\|}\nabla f(x_k)
$$

\textbf{where}
$$
\tau_k=\left\{
\begin{aligned}
	1,\quad \text{if } \nabla f(x_k)^TB_k\nabla f(x_k)\leq 0\\
	\min\{\|\nabla f(x_k)\|^3/(\nabla f(x_k)^TB_k\nabla f(x_k)),1\},\quad \text{otherwise.}
\end{aligned}
\right.
$$
\textbf{解:}取$p_k^s=-\frac{\Delta_k}{\|\nabla f(x_k)\|}\nabla f(x_k)$,则有$q_k(\tau p_k^s)=\frac{1}{2}\tau^2{p_k^s}^TB_kp_k^s+\tau \nabla f(x_k)^Tp_k^s+f(x_k)$。

下面分情况进行讨论。

(1)如果$\nabla f(x_k)^TB_k\nabla f(x_k)\leq0$,由函数图像可知,在$\tau>0$时随$\tau_k$增大,$q_k(\tau_k p_k^s)$的值减小,因此$\tau_k$取不超过信赖域范围内最大值$1$时,$q(p_k^c)$取最小值。

(2)如果$\nabla f(x_k)^TB_k\nabla f(x_k)>0$,则$\tau_0=-\frac{\nabla f(x_k)^Tp_k^s}{{p_k^s}^TB_kp_k^s}$是$q_k(\tau p_k^s)$关于$\tau$的最小值点。

(i)当$p_k^c$在信赖域范围内时,$\tau_0\leq 1$,此时$\tau_k=\tau_0$就是我们需要的值。化简$\tau_0$可知,此时有$$\tau_k=\frac{\|\nabla f(x_k)\|^3}{\Delta_k\nabla f(x_k)^TB_k\nabla f(x_k)}$$

(ii)当$p_k^c$在信赖域范围内时,$\tau_0> 1$,此时根据函数图像可知$q_k(\tau p_k^s)$在$\tau\in[0,\tau_0]$范围内下降,又$\tau\in[0,1]$,故$\tau_k$取$1$时是我们的目标点。

综上所述,$\tau_k=\left\{
\begin{aligned}
	1,\quad \text{if } \nabla f(x_k)^TB_k\nabla f(x_k)\leq 0\\
	\min\{\frac{\|\nabla f(x_k)\|^3}{\Delta_k\nabla f(x_k)^TB_k\nabla f(x_k)},1\},\quad \text{otherwise.}
\end{aligned}
\right.$,原命题得证。
\end{homeworkProblem}

\begin{homeworkProblem}
\textbf{3. If symmetric $B\in R^{n\times n}$ has factorization $B=Q\Lambda Q^T$ where $Q=(q_1,q_2,\cdots,q_n)$ is orthogonal, $\Lambda=diag(\lambda_1,\lambda_2,\cdots,\lambda_n)$. Try to prove}
$$
\left\{
\begin{aligned}
	(B+\lambda I)p & =  -g \\
	\|p\| & =  \Delta \\
\end{aligned}
\right.
$$

\textbf{solution can be expressed by}
$$
p(\lambda)=-\sum_{i=1}^{n}\frac{q_i^Tg}{\lambda_i+\lambda}q_i.
$$

\textbf{Furthermore, try to prove}
$$
\frac{d}{d\lambda}(\|p(\lambda)\|^2)=-2\sum_{i=1}^{n}\frac{(q_i^Tg)^2}{(\lambda_i+\lambda)^3}
$$
\textbf{解:}我们将$p(\lambda)=-\sum_{i=1}^{n}\frac{q_i^Tg}{\lambda_i+\lambda}q_i$代入$(B+\lambda I)p=-g$后得到
$$
\begin{aligned}
(B+\lambda I)p&=\sum_{i=1}^n(\lambda_i q_i q_i^T+\lambda e_ie_i^T)(-\sum_{i=1}^{n}\frac{q_i^Tg}{\lambda_i+\lambda}q_i)\\
&=-\sum_{i=1}^n(\lambda_i q_i q_i^T+\lambda e_ie_i^T)\sum_{i=1}^{n}\frac{q_i^Tg}{\lambda_i+\lambda}q_i\\
&=-(\sum_{i=1}^{n}\frac{\lambda_iq_iq_i^Tq_iq_i^Tg}{\lambda_i+\lambda}+\lambda E \sum_{i=1}^{n}\frac{q_iq_i^Tg}{\lambda_i+\lambda})\\
&=-(\sum_{i=1}^n\frac{\lambda_iq_iq_i^Tg}{\lambda_i+\lambda}+\lambda\sum_{i=1}^n\frac{q_iq_i^Tg}{\lambda_i+\lambda})\\
&=-\sum_{i=1}^{n}q_iq_i^Tg\\
&=-QEQ^Tg=-g
\end{aligned}
$$

因为上述过程是可逆的,则说明$p(\lambda)=-\sum_{i=1}^{n}\frac{q_i^Tg}{\lambda_i+\lambda}q_i$是方程组的解,这里我们并未用到$\|p\| = \Delta$的条件。

进一步地,有
$$
\begin{aligned}
\|p(\lambda)\|^2&=p(\lambda)^T\cdot p(\lambda)\\
&=\sum_{i=1}^{n}\frac{q_i^Tg}{\lambda_i+\lambda}q_i^T\cdot \sum_{i=j}^{n}\frac{q_j^Tg}{\lambda_j+\lambda}q_j^T\\
&=\sum_{i,j}\frac{(q_i^Tg)(q_j^Tg)}{(\lambda_i+\lambda)(\lambda_j+\lambda)}\delta_{i,j}\\
\end{aligned}
$$
其中
$$\delta_{i,j}=\left\{
\begin{aligned}
	1,&\quad i=j\\
	0,&\quad i\neq j
\end{aligned}
\right.
$$
因此有
$$
\|p(\lambda)\|^2=\sum_{i=1}^{n}\frac{(q_i^Tg)^2}{(\lambda_i+\lambda)^2}
$$
求导得到$$\frac{d}{d\lambda}(\|p(\lambda)\|^2)=-2\sum_{i=1}^{n}\frac{(q_i^Tg)^2}{(\lambda_i+\lambda)^3}$$
\end{homeworkProblem}

\begin{homeworkProblem}
\textbf{4. Let $p_k=\arg\min\{m_k(p):\|p\|\leq\Delta_k,s\in {span}[g_k,B_k^{-1}g_k]\}$, where $m_k(p)=f(x_k)+g_k^Tp+\frac{1}{2}p^TB_kp$, $B_k$ is symmetric positive. Try to find the explicit solution $p_k$.}

\textbf{解:}

如果$p_B=-B_k^{-1}g_k$并且$\|p_B\|\leq\Delta_k$,则$p_k=p_B=-B_k^{-1}g_k$。

如果$\|p_B\|>\Delta_k$,令$p=p^U+(1-\tau)(p^U-p^B)$,$\tau\in[1,2]$,其中
$$
p^U=-\frac{g_k^Tg_k}{g_k^TB_kg_k}g_k
$$

因此$p(\tau)=-\frac{g_k^Tg_k}{g_k^TB_kg_k}g_k+(1-\tau)(B_k^{-1}g_k-\frac{g_k^Tg_k}{g_k^TB_kg_k}g_k)$,显然$\|p^U+(1-\tau)(p^U-p^B)\|$随$\|(1-\tau)(p^U-p^B)\|$增大而增大。

当$\tau\in[1,2]$增大时,$\|p\|=\|p^U+(1-\tau)(p^U-p^B)\|$增大,下面证明$m_k(p(\tau))$随$\tau$增大而减小。

$$
\begin{aligned}
m_k(p(\tau))&=m_k(p^U+(1-\tau)(p^U-p^B))\\
&=f+g^Tp+\frac{1}{2}p^TBp\\
&=f+g^Tp^U+(\tau-1)[g^T(p^B-p^U)+\frac{1}{2}{p^U}^TB(p^B-p^U)\\
&\quad+\frac{1}{2}(p^B-p^U)^TBp^U)+(\tau-1)^2\frac{1}{2}(p^B-p^U)^TB(p^B-p^U)]\\
\end{aligned}
$$

令$$\tau_0=\frac{g^T(p^B-p^U)+\frac{1}{2}{p^U}^TB(p^B-p^U)+\frac{1}{2}(p^B-p^U)^TBp^U}{(p^B-p^U)^TB(p^B-p^U)}$$

则$m_k(p(\tau))$会在$\tau<\tau_0$时下降,在$\tau>0$时增大。

在无约束条件下,$\tilde{\tau}=2$(此时$p(\tilde{\tau})=p^B$)是全局最小值点,这说明$\tau_0=2=\tilde{\tau}$,故$\tau\in[1,2]$时$m_k(p(\tau))$随$\tau$增大而减小。

在$\|p\|\leq\Delta_k,s\in {span}[g_k,B_k^{-1}g_k]$的约束条件下,$p_k$的显式解由$\|p^U+(1-\tau)(p^U-p^B)\|=\Delta_k^2$决定,这是因为$\tau\in[1,2]$时$\|p(\tau)\|$增大而$m(p(\tau))$减小。

综上,有
$$
p_k=\left\{
\begin{aligned}
	-B_k^{-1}g_k,&\quad \|-B_k^{-1}g_k\|\leq\Delta_k\\
	-\frac{g_k^Tg_k}{g_k^TB_kg_k}g_k,&\quad \|-B_k^{-1}g_k\|>\Delta_k
\end{aligned}
\right.
$$
\end{homeworkProblem}

\end{document}
